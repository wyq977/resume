\documentclass[10pt,a4paper]{article}
\usepackage[a4paper,margin=0.5in]{geometry}
\usepackage[utf8]{inputenc}
\usepackage{mdwlist}
\usepackage{hyperref}
\usepackage{fix-cm}
\usepackage[T1]{fontenc}
\usepackage{textcomp}
\usepackage{tgpagella}
\usepackage[
backend=biber,
style=mla,
sorting=ynt,
maxbibnames=99
]{biblatex}
\addbibresource{publications.bib}


\pagestyle{empty}
\setlength{\tabcolsep}{0em}

% indentsection style, used for sections that aren't already in lists
% that need indentation to the level of all text in the document
\newenvironment{indentsection}[1]%
{\begin{list}{}%
	{\setlength{\leftmargin}{#1}}%
	\item[]%
}
{\end{list}}

% opposite of above; bump a section back toward the left margin
\newenvironment{unindentsection}[1]%
{\begin{list}{}%
	{\setlength{\leftmargin}{-0.5#1}}%
	\item[]%
}
{\end{list}}

% format two pieces of text, one left aligned and one right aligned
\newcommand{\headerrow}[2]
{\begin{tabular*}{\linewidth}{l@{\extracolsep{\fill}}r}
	#1 &
	#2 \\
\end{tabular*}}

% make "C++" look pretty when used in text by touching up the plus signs
\newcommand{\CPP}
{C\nolinebreak[4]\hspace{-.05em}\raisebox{.22ex}{\footnotesize\bf ++}}

% and the actual content starts here
\begin{document}


%%%%%%%%%%%%%%%%%%%%%%%%%%%%%
% NAME AND PERSONAL DETAILS %
%%%%%%%%%%%%%%%%%%%%%%%%%%%%%

\begin{center}
{\LARGE \textbf{Yongqi Wang}}

% \ \ Zürich, Switzerland \\
% \textbullet \ \ \href{tel:410782510650}{+41 78 251 06 50} \ \
% \textbullet \ \ \href{mailto:wangyq977@gmail.com}{wangyq977@gmail.com}
\end{center}


%%%%%%%%%%%%%%%%
% 1. EDUCATION %
%%%%%%%%%%%%%%%%

\hrule
\vspace{-1em}
\subsection*{Education}

\begin{itemize}
	\parskip=0.1em

	\item
	\headerrow
		{\textbf{Centrum Wiskunde \& Informatica (CWI)}}
		{\textbf{Amsterdam, Netherlands}}
	\\
	\headerrow
		{\emph{PhD, Machine Learning, supervised by Prof. dr. \href{https://safestatistics.com/}{Peter Gr\"{u}nwald}}}
		{\emph{Dec 2024 - Present}}

	\item
	\headerrow
		{\textbf{Eidgenössische Technische Hochschule Zürich (ETHz)}}
		{\textbf{Zürich, Switzerland}}
	\\
	\headerrow
		{\emph{MSc, Computational Biology and Bioinformatics}}
		{\emph{Sep 2018 - Jun 2023}}

	\item
	\headerrow
		{\textbf{The Hong Kong Polytechnic University (PolyU)}}
		{\textbf{Hong Kong, China}}
	\\
	\headerrow
		{\emph{BSc (Hons), Applied Biology with Biotechnology, Minor in Appliled Mathematics}}
		{\emph{Sep 2014 - May 2018}}

	\item
	\headerrow
		{\textbf{The University of Waterloo (UWaterloo)}}
		{\textbf{Waterloo, Canada}}
	\\
	\headerrow
		{\emph{Exchange}}
		{\emph{Jan 2017 - May 2017}}

\end{itemize}


%%%%%%%%%%%%%%%%%%%%%%%%
% 2. EXPERIENCE - WORK %
%%%%%%%%%%%%%%%%%%%%%%%%

\hrule
\vspace{-1em}
\subsection*{Experience}

\begin{itemize}
	\parskip=0.1em

	\item
	\headerrow
		{\textbf{ETH Zürich}}
		{\textbf{Zürich, Switzerland}}
	\\
	\headerrow
		{\emph{Master's Thesis @ Prof. dr. \href{https://stat.ethz.ch/~vsara/}{Sara van de Geer}}}
		{\emph{Oct 2022 - Mar 2023}}
	\begin{itemize*}
		\item Surveyed the approximation properties of two-layer neural networks (2NN)
		\item Characterized function spaces where 2NN is effective in approximation
		% \item Drew connections between the function spaces associated with 2NN
	\end{itemize*}

	\item
	\headerrow
		{\textbf{ETH Zürich}}
		{\textbf{Zürich, Switzerland}}
	\\
	\headerrow
		{\emph{Lab Rotation @ \href{https://bsse.ethz.ch/cobi}{CoBi}}}
		{\emph{Oct 2020 - Dec 2020}}
	\begin{itemize*}
    	\item Adapted 2D cellular simulation framework (\href{https://tanakas.bitbucket.io/lbibcell/index.html}{LBIBCell}) for morphogen gradient detection	
		% \item Added various boundary conditions in computational fluid simulation in \href{https://tanakas.bitbucket.io/lbibcell/index.html}{LBIBCell}
		\item Parameter screening for viable synthetic tissues on Euler cluster
	\end{itemize*}

	\item
	\headerrow
		{\textbf{ETH Zürich}}
		{\textbf{Zürich, Switzerland}}
	\\
	\headerrow
		{\emph{Lab Rotation @ \href{https://bsse.ethz.ch/cobi}{CoBi}}}
		{\emph{Apr 2020 - Jul 2020}}
	\begin{itemize*}
    	\item Benchmarked different 3D surface re-meshing algorithms for complex cellular structures
		\item Implemented Poisson disc surface sampling and re-meshing in \CPP (VTK)
		% \item Integrated the surface re-meshing, IO (vtk) in the simulation framework 
	\end{itemize*}

	\item
	\headerrow
		{\textbf{Universität Zürich}}
		{\textbf{Zürich, Switzerland}}
	\\
	\headerrow
		{\emph{Lab Rotation @ \href{http://bachlab.org/}{bachlab}}}
		{\emph{Apr 2019 - Jul 2019}}
	\begin{itemize*}
		\item Built a \href{https://github.com/fmelinscak/cognibench}{benchmark framework} for cognitive models with \href{https://github.com/scidash/sciunit}{SciUnit} in Python 
		\item Added a CI/CD pipeline and distribution for the Python package
	\end{itemize*}

	\item
	\headerrow
		{\textbf{Hong Kong Polytechnic University}}
		{\textbf{Hong Kong, China}}
	\\
	\headerrow
		{\emph{Research Assistant @ \href{https://www.polyu.edu.hk/en/abct/people/academic-staff/dr-ko-cb-ben/}{Dr Ko Chi-bun, Ben}}}
		{\emph{Feb 2018 - Jul 2018}}
	\begin{itemize*}
		\item Classified protein binding patterns in ChIP-seq data
    	\item Identified differential binding events and potential gene targets % via \texttt{DESeq2}
    	% \item Visualization to facilitate graphical representation of the medical data in Python, R % by \texttt{matplotlib} (Python), Shiny in R
	\end{itemize*}

	\item
	\headerrow
		{\textbf{China Agricultural University}}
		{\textbf{Beijing, China}}
	\\
	\headerrow
		{\emph{Research Assistant @ \href{https://cvm.cau.edu.cn/art/2011/12/26/art_3747_40.html}{Prof. Xun Suo}}}
		{\emph{Apr 2016 - Aug 2016}}
	\begin{itemize*}
		\item Built and maintained an internal server for microarray analysis
		\item Versioned \textsf{R}/Bioconductor packages for common analysis workflows
	\end{itemize*}

\end{itemize}


%%%%%%%%%%%%%%%%%%%%%%%%%%%%
% 3. EXPERIENCE - PROJECTS %
%%%%%%%%%%%%%%%%%%%%%%%%%%%%

% \hrule
% \vspace{-1em}
% \subsection*{Experience - \href{https://github.com/wyq977}{Projects}}

% % \vspace{-0.4em}

% \begin{itemize*}
% 	\parskip=0.1em

% 	% \item Helped integrate a CI/CD toolbox for benchmarking behavioural data in \href{http://bachlab.org/}{bachlab}@UZH.

% 	\item Finished multiple course-work projects such as object detection with ML, texture extraction.

% 	\item Participant in International Olympiad selection camps (Biology) in Guangdong, China
	
% \end{itemize*}


%%%%%%%%%%%%%
% 4. AWARDS %
%%%%%%%%%%%%%

\hrule
\vspace{-1em}
\subsection*{Awards}

\begin{itemize}
	\parskip=0.1em
		\item
		\headerrow
			{Dean’s List of Outstanding Students, Faculty of Applied Science and Textiles, \textbf{PolyU}}
			{2017}

		% \item
		% \headerrow
		% 	{Work-Integrated Education Offshore Sponsorship, PolyU}
        %     {2017}

        % \item
		% \headerrow
        %     {Wong Tit-shing Student Exchange Scholarship, PolyU}
        %     {2017}

\end{itemize}


%%%%%%%%%%%%%%%%%%%%
% 5a. PUBLICATIONS %
%%%%%%%%%%%%%%%%%%%%

\hrule
\vspace{-1em}
\subsection*{Publications}

\nocite{*} % This includes all entries in the bibliography file

\printbibliography[heading=none]


% \begin{refsection}
% 	\section*{Publications}
% 	\nocite{*}                          % cite everything
% 	\printbibliography[heading = none,  % no heading (e.g., "References")
% 	keyword = article,                  % FILTER BY < article > KEYWORD
% 	env = mybib]                        % Use mybib style
% \end{refsection}

%%%%%%%%%%%%%%%%%%%%%%%
% 5. TECHNICAL SKILLS %
%%%%%%%%%%%%%%%%%%%%%%%

\hrule
\vspace{-1em}
\subsection*{Technical Skills}

\begin{indentsection}{\parindent}
\hyphenpenalty=1000
\begin{description*}
	% \item 
	\item {Python, R, Shell, \CPP, \LaTeX, git, Docker}
	% {\textbf{Languages}}

	% \begin{itemize*}
    %     \item {Python, R, Bash, \CPP, Google Cloud Platform}
	% \end{itemize*}

\end{description*}



\end{indentsection}


%%%%%%%%%%%%%%%%%%%%
% 6. COURSES TAKEN %
%%%%%%%%%%%%%%%%%%%%

% \hrule
% \vspace{-1em}
% \subsection*{Courses taken}

% \begin{indentsection}{\parindent}
% \hyphenpenalty=1000
% \begin{description*}
% 	\item
% 	{\textbf{Statistics}}
% 	Mathematical Statistics, Probability Theory, Casualty, Empirical Process Theory, Statical Models in Computational Biology, Mathematical Tools in ML, High-Dimensional Statistics

% 	\item
% 	{\textbf{Others}}
% 	Data Mining, Intro to ML, Computational Intelligence Lab, Advanced Machine learning

% \end{description*}
% \end{indentsection}

\end{document}
